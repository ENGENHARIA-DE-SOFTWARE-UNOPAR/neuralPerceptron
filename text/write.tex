
\section{Introdução}

\noindent \begin{minipage}[c]{0.6\textwidth}
  \vspace {1cm}
  \begin{description}
    \item [Banana] Exemplo de mini página com figura e seus respectivos rotulos, para que sejam referenciados ao decorrer do texto.
    \item [Maça] Veja que a Figura \ref{fig:logo_weka}, está reservando um espaço para adição de figuras, e o mesmo já esta referenciando seu autor e sua nomeclatura com o indice automatico.
  \end{description}

\end{minipage}
\begin{minipage}[c]{0.4\textwidth}

  \includegraphics[width=\textwidth]{figure/weka-logo.jpg}
  	\label{fig:logo_weka}
    \captionof{figure}{Weka, \cite{Weka:2023}}
    %\captionof*{figure}{Fonte: \citeonline{linux:2023}}
\end{minipage}


\section{Métodos}


\begin{figure}[H] %Figuras da aula pratica 1.1
  \center
  \subfigure[Perceptron.\label{fig:frist}]{\includegraphics[scale=0.6]{figure/result.png}}

  \subfigure[Perceptron 75\%.\label{fig:75}]{\includegraphics[scale=0.6]{figure/result_75p.png}}
  \caption{Rede neural Perceptron, O autor}\label{fig:redeNeural}
\end{figure}

%%%%%%%%%%%%%%%%%%%%%%%%%%%%%%%%%%%%%%%%%%%%%%%%%%%%%%%%%%






Aqui é um exemplo de rodapé. \footnote{Um exemplo de rodapé}

\begin{equation}
\int_{-L}^{L} sen \frac{m \pi x}{2}\,sen \frac{n \pi x}{2}\,dx =
\left \{
\begin{array}{cc}
0, & m \neq n \\
1, & m = n \\
\end{array}
\right.
\end{equation}




\begin{enumerate}[label=\Roman{*}, ref=(\roman{*})]
  \item fsfsdf
  \item kugfhiuh
\end{enumerate}

\begin{asparaenum}
\item Anterior ... \cite{ninguem2022curioso}
\item Próximo ... \label{pl1}
\end{asparaenum}


\section{Resultados}







\section{Conclusões}




  %$X \xLongleftarrow[\text{NATAN}]{\text{OGLIARI}} Y $ %COM TEXTO
	% $\uparrow$ %Seta para Cima
	%$\overleftarrow{NATAN}$
